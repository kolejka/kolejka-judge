\subsection{CMake}

\begin{minted}[linenos]{python}
checking.add_steps(
    cmake=CMake(),
    make=Make(build_dir='build'),
    solution=RunSolution(executable='build/untitled', stdout='out'),
    diff=Diff(),
)
status_code, detailed_result = checking.run()
\end{minted}

\code{CMake} without arguments sets the source directory to the current directory, and a build directory to
\code{build}.
\code{Make} receives the build directory name as the argument, since its default is set to the current directory.
Then, because the project name in \code{CMakeLists.txt} is set to \code{untitled}, the \code{Makefile} generated via the
\shell{cmake} command builds the executable named \code{untitled}.
Path to this executable is then passed to the \code{RunSolution}, and the \shell{stdout} is redirected to the file
called \shell{out}.
Finally, the \code{Diff} command -- which defaults to \shell{wzo} and \shell{out} as the file names -- compares the
output with the expected answer.
The \shell{wzo} file needs to be provided by the test case, just as the source files required by \code{CMakeLists.txt},
and the \code{CMakeLists.txt} file itself.
