\section{Checking}\label{sec:checking}
    \python{Checking} class's responsibility is to provide a readable and clean interface for binding together the two
    abstractions: \hyperref[sec:steps]{Steps} and \hyperref[sec:environments]{Environments}.

\subsection*{\python{Checking}}\label{subsec:Checking}

\begin{itemize}[label={}]
    \item \docfunc{__init__(self, environment)}
        \docfuncdesc{
            Initializes the instance with a given environment, which will be used during the
            \hyperref[sec:Checking.run]{\python{run()}} call.
        }

    \item \phantomsection \label{sec:add_steps} \docfunc{add_steps(self, *args, **kwargs)}
        \docfuncdesc{
            Takes the steps that will be run during the
            \hyperref[sec:Checking.run]{\python{run()}} call as the arguments.
            If the arguments are passed as \python{kwargs}, the argument names will be used as step names in the
            execution details and logs.
            Otherwise, consecutive integers will be assigned as step names.
            If a step with a given name was already added (i.e. during a previous \python{add_steps()} call),
            \code{TypeError} will be raised.
        }

    \item \phantomsection \label{sec:Checking.run} \docfunc{run(self)}
        \docfuncdesc{
            Calls the \hyperref[sec:run_steps]{\python{run_steps()}} method on \python{self.environment} with the
            previously registered steps.
        }

    \item \docfunc{format_result(self, result)}
        \docfuncdesc{
            Takes the result of \hyperref[sec:Checking.run]{\python{run()}} as an argument, and returns a JSON-compatible
            dictionary using the \hyperref[sec:format_execution_status]{\python{format_execution_status()}} method on
            \python{self.environment}.
        }
\end{itemize}
