\section{Basic structure}\label{sec:example}

\usemintedstyle{tango}
\begin{minted}[linenos]{python}
checking = Checking(environment=detect_environment())
checking.add_steps(
    compile=CompileCpp('solution.cpp'),
    run_solution=RunSolution(stdout='out'),
    diff=Diff(),
)
status_code, detailed_result = checking.run()
\end{minted}

We can distinguish three key parts:
\begin{itemize}
    \item Initialization (line 1)

    The only argument to the \python{Checking} constructor is the environment responsible for running
    the subsequent tasks.
    \hyperref[sec:detect_environment]{\python{detect_environment()}} is an utility function which automatically detects
    the environment based on the command line arguments.
    For more information refer to the \hyperref[sec:environments]{Environments} section.
    \item Configuration (lines 2--6)

    This section is responsible for defining the judge pipeline.
    The steps will be executed in the same order as they were specified in the calls' arguments.
    The usage of keyword over regular arguments in \hyperref[sec:add_steps]{\python{add_steps()}} is purely optional, designed to provide an
    option to specify the custom step identifiers.
    \item Launching the checker (line 7)

    The \hyperref[sec:Checking.run]{\python{run()}} method runs the configured pipeline, and returns a 2-tuple, consisting of the final
    judge verdict code, and a detailed log of the steps ran.
\end{itemize}
