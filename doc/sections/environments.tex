\section{Environments}\label{sec:environments}
    Environments define how the steps are run, what limits are applied to them, and how the processes are monitored.
    They are also responsible for implementing various checks that validators might want to use, but their exact
    implementation would depend on the environment's specification.\\

    There are three predefined environments, \hyperref[sec:LocalComputer]{\python{LocalComputer}},
    \hyperref[sec:PsutilEnvironment]{\python{PsutilEnvironment}} and
    \hyperref[sec:KolejkaObserver]{\python{KolejkaObserver}}.\\

    \hyperref[sec:LocalComputer]{\python{LocalComputer}} is supposed to provide a minimal support for running the judge
    without installing any additional packages (provided \shell{/usr/bin/time} is available).
    It doesn't support any limits, and should be used solely for debugging/testing purposes.\\

    \hyperref[sec:PsutilEnvironment]{\python{PsutilEnvironment}} is an slightly enhanced environment, which uses the
    \shell{psutil} module and a separate thread to continuously query the step being run for used resources.\\

    \hyperref[sec:KolejkaObserver]{\python{KolejkaObserver}} uses the \shell{kolejka-observer} package, and is
    recommended for any serious checking systems, as it provides the greatest flexibility in terms of the available
    limits.\\

    The utility function \hyperref[sec:detect_environment]{\python{detect_environment()}} can be used to
    automatically select the environment, based on the command line arguments.\\

    In order to create an own environment, its implementation has to implement all abstract methods, i.e.
    \hyperref[sec:run_command]{\python{run_command()}} and
    \hyperref[sec:format_execution_status]{\python{format_execution_status()}}.\\

\subsection*{\python{ExecutionEnvironment}}

\renewcommand\labelitemiii{\textbullet}

\begin{itemize}[label={}]
    \item This is the base class, defining common methods for the environments.

    \item \docfunc{__init__(self, output_directory)}
        \docfuncdesc{
            Constructor, which takes as an argument the directory where all files created by the checking run will be
            stored.
            Commands will have this directory set as a current working directory.
            Note that the executed programs aren't limited by this setting when they don't follow the working directory,
            especially if they use absolute paths.\\
            The \hyperref[sec:Validators]{\python{Validators}} nested class is instantiated with the environment as an
            argument and assigned to a variable, creating a circular dependency between these two.
        }

    \item \docfunc{set_limits(self, **kwargs)}
        \docfuncdesc{
            Filters the limits passed as the arguments, based on the \python{self.recognized_limits} variable, and prints out a warning on \shell{stderr} for each unrecognized
            one.
            The identified limits are saved to be used during the following \hyperref[sec:run_command]{
                \python{run_command()}} calls,
            until the next \python{set_limits()} invocation.
        }

    \item \phantomsection \label{sec:run_steps} \docfunc{run_steps(self, steps)}
        \docfuncdesc{
            Executes the received steps one by one, halting immediately when any of the steps returns an exit status
            (e.g. \shell{CME}).
            Returns a 2-tuple containing the final status (either one of the exit statuses, or \shell{OK}),
            and a dictionary with execution statistics for each ran step.
        }

    \item \phantomsection \label{sec:run_command_step} \docfunc{run_command_step(self, step, name)}
        \docfuncdesc{
            Responsible for running the command step, which consists of the following parts:
            \begin{itemize}
                \item verifying that step is configured correctly
                \item verifying the prerequisites are met
                \item setting the limits requested by the step (see
                      \hyperref[sec:get_limits]{\python{CommandBase.get_limits()}})
                \item evaluating the \hyperref[sec:DependentExpr]{\python{DependentExpr}} expressions
                \item calling the \hyperref[sec:run_command]{\python{run_command()}} method
                \item checking the postconditions
                \item restoring the old limits
            \end{itemize}
        }

    \item \phantomsection \label{sec:run_command}
        \docfunc{run_command(self, command, stdin, stdout, stderr, env, user, group)}
        \docfuncdesc{
            Abstract method.
            Responsible for running the specified command within the appropriate launch configuration, consisting of
            standard input/output files (handles opened from \python{stdin}, \python{stdout}, \python{stderr} arguments,
            all of type \python{pathlib.Path}),
            environment variables (\python{env}), process permissions (\python{user} and \python{group}) and limits
            (from previous \python{set_limits()} call).
            Returns the optional \hyperref[sec:ExecutionStatistics]{execution statistics} object.
        }

    \item \phantomsection \label{sec:run_task_step} \docfunc{run_task_step(self, step, name)}
        \docfuncdesc{
            Responsible for running the task step, which consists of the following parts:
            \begin{itemize}
                \item verifying the prerequisites are met
                \item calling the \hyperref[sec:execute]{\python{execute()}} method
            \end{itemize}
        }

    \item \phantomsection \label{sec:env_get_env} \docfunc{get_env(self)}
        \docfuncdesc{
            Returns the dictionary of environment variables that will be passed to the spawned process.
            Steps can expand and modify the mapping by overriding the
            \hyperref[sec:command_get_env]{\python{CommandBase.get_env()}} method.
        }

    \item \phantomsection \label{sec:set_variable} \docfunc{set_variable(self, variable_name, value)}
        \docfuncdesc{
            Used by the tasks to store any value that should be accessible by the command steps, but is
            undetermined before run-time.
        }

    \item \phantomsection \label{sec:format_execution_status} \docfunc{format_execution_status(cls, status)}
        \docfuncdesc{
            Abstract classmethod.
            Responsible for serializing the execution statistics data into a dictionary containing solely
            JSON-compatible types.
            Useful for logging.
        }

    \item \docfunc{get_path(self, path)}
        \docfuncdesc{
            Responsible for returning a path uniquely determined by the argument, that is a subdirectory of
            \python{self.output_directory}.
        }

    \item \docfunc{get_file_handle(file, mode)}
        \docfuncdesc{
            Creates all required parent directories of \python{file}, if they don't exist.
            Returns a file handle opened with the specified mode.
        }

    \item \phantomsection \label{sec:Validators} \docfunc{Validators}
        \docfuncdesc{
            Class specifying the environment-specific validators, available for use mainly in the prerequisites.
            When requesting an unknown validator, an no-op function is returned instead, to ensure maximum compatibility
            while switching between multiple environments.
        }

\end{itemize}

\subsection*{\python{LocalComputer}}\label{sec:LocalComputer}

\begin{itemize}[label={}]
    \item \python{recognized_limits = []}

    \item \docfunc{run_command(self, command, stdin, stdout, stderr, env, user, group)}
        \docfuncdesc{
            Runs the command using the \shell{/usr/bin/time} tool to measure the time and memory used.
            Returns the \hyperref[sec:LocalComputer.LocalStats]{\python{LocalComputer.LocalStats}} object containing
            execution statistics.
        }

    \item \docfunc{format_execution_status(cls, status)}
        \docfuncdesc{
            Implements the \hyperref[sec:format_execution_status]{
                \python{ExecutionEnvironment.format_execution_status()}
            }
            method.
        }

    \item \phantomsection \label{sec:LocalComputer.LocalStats} \docfunc{LocalStats}
        \docfuncdesc{
            Object representing the execution statistics.
            Contains three properties: \code{time}, \code{memory} and \code{cpus}.
        }

    \item \docfunc{Validators}
        \docfuncdesc{
            Inherits all validators from the \hyperref[subsec:LocalExecutionEnvironmentValidatorsMixin]{
            \python{LocalExecutionEnvironmentValidatorsMixin}}.
        }
\end{itemize}

\subsection*{\python{PsutilEnvironment}}\label{sec:PsutilEnvironment}

\begin{itemize}[label={}]
    \item \python{recognized_limits = ['cpus', 'cpus_offset', 'time', 'memory']}

    \item \docfunc{run_command(self, command, stdin, stdout, stderr, env, user, group)}
        \docfuncdesc{
            Runs the command using the \python{psutil.Popen} function, and starts a separate thread monitoring the
            resource usage of the launched process (see \hyperref[sec:monitor_process]{\python{monitor_process()}}).
            Returns the \hyperref[sec:PsutilEnvironment.LocalStats]{\python{PsutilEnvironment.LocalStats}} object containing
            execution statistics.
        }

    \item \phantomsection \label{sec:monitor_process} \docfunc{monitor_process(self, process, execution_status)}
        \docfuncdesc{
            Sets the \python{cpu_affinity} limit on the process passed as an argument, then proceeds to query the
            time and memory usage each 0.1s.
            Kills the process if it exceeds the \python{time} or \python{memory} limits.
            After the program finishes its execution, sets the gathered statistics on the \python{execution_status}
            argument.
        }

    \item \docfunc{format_execution_status(cls, status)}
        \docfuncdesc{
            Implements the \hyperref[sec:format_execution_status]{\python{ExecutionEnvironment.format_execution_status}}
            method.
        }

    \item \phantomsection \label{sec:PsutilEnvironment.LocalStats} \docfunc{LocalStats}
        \docfuncdesc{
            Object representing the execution statistics.
            Contains three properties: \code{time}, \code{memory} and \code{cpus}.
        }

    \item \docfunc{Validators}
        \docfuncdesc{
            Inherits all validators from the \hyperref[subsec:LocalExecutionEnvironmentValidatorsMixin]{
            \python{LocalExecutionEnvironmentValidatorsMixin}}.
        }
\end{itemize}

\subsection*{\python{KolejkaObserver}}\label{sec:KolejkaObserver}

\begin{itemize}[label={}]
    \item \python{recognized_limits = ['cpus', 'cpus_offset', 'pids', 'memory', 'time']}

    \item \docfunc{run_command(self, command, stdin, stdout, stderr, env, user, group)}
        \docfuncdesc{
            Runs the command using the \python{observer.run()} function from the \python{kolejka-observer} package.
            Returns an enriched \python{CompletedProcess} object, containing
            execution statistics.
        }

    \item \docfunc{format_execution_status(cls, status)}
        \docfuncdesc{
            Implements the \hyperref[sec:format_execution_status]{
                \python{ExecutionEnvironment.format_execution_status()}
            }
            method.
        }

    \item \docfunc{Validators}
        \docfuncdesc{
            Inherits all validators from the \hyperref[subsec:LocalExecutionEnvironmentValidatorsMixin]{
            \python{LocalExecutionEnvironmentValidatorsMixin}}.
        }
\end{itemize}

\subsection*{\docfunc{detect_environment()}}\label{sec:detect_environment}

\begin{itemize}[label={}]
    \item Returns the environment based on the command line arguments.
    \item \shell{--local} (\textit{default}) -- \hyperref[sec:LocalComputer]{\python{LocalComputer}}
    \item \shell{--psutil} -- \hyperref[sec:PsutilEnvironment]{\python{PsutilEnvironment}}
    \item \shell{--kolejkaobserver} -- \hyperref[sec:KolejkaObserver]{\python{KolejkaObserver}}
    \item Remaining arguments are then passed to the environment-specific parsers and, if recognized, to the
          environment constructor as \python{kwargs}.
\end{itemize}

\subsection*{\docfunc{LocalExecutionEnvironmentValidatorsMixin}}\label{subsec:LocalExecutionEnvironmentValidatorsMixin}

\begin{itemize}[label={}]
    \item Defines the methods that are shared between all environments running on a local file system.
          Currently two validators are implemented:
    \item \python{file_exists(self, file)} - checks if the file exists in the file system
    \item \python{program_exists(self, file)} - checks if the program exists in the file system
\end{itemize}
