\section{Environments}\label{sec:environments}

    There are two environments, \hyperref[sec:LocalComputer]{\python{LocalComputer}} and
    \hyperref[sec:KolejkaObserver]{\python{KolejkaObserver}}.
    The first one is supposed to provide a minimal support for running the judge without installing and launching
    kolejka-observer.
    It doesn't support any limits, and should be used solely for debugging purposes.
    \hyperref[sec:KolejkaObserver]{\python{KolejkaObserver}} should be used in the real situations instead.
    The utility function \hyperref[sec:detect_environment]{\python{detect_environment()}} can be used to
    automatically select the environment,
    based on the command line arguments.

\section*{\python{ExecutionEnvironment}}

% TODO: in doc.tex?
\renewcommand\labelitemiii{\textbullet}

\begin{itemize}[label={}]
    \item This is the base class, defining common methods for the environments.
    \item \docfunc{set_limits(self, **kwargs)}
        \docfuncdesc{
            Filters out the unrecognized limits passed as the arguments, based on the \\
            \python{self.recognized_limits} variable.
            Warning about the unknown limits is then printed on the \python{stderr}.
            Finally the identified limits are saved to be used during the following \python{run_command()} calls,
            until the next \python{set_limits()} invocation.
        }

    \item \docfunc{run_command(self, command, stdin, stdout, stderr, env)}
        \docfuncdesc{
            Abstract method.
            Responsible for running the specified command within the appropriate launch configuration, consisting of
            standard input/output files (handles opened from \python{stdin}, \python{stdout}, \python{stderr} arguments,
            all of type \python{pathlib.Path}),
            environment variables (\python{env}), and limits (from previous \python{set_limits()} call).
            Returns the execution statistics object, details of which are left up to the particular environment -
            some attributes are, however, required for all of the environments (see TODO \python{CompletedProcess}).
        }

    \item \docfunc{run_step(self, step, name)}
        \docfuncdesc{
            Responsible for running the step, which consists of the following parts:
            \begin{itemize}
                \item verifying that step is configured correctly
                \item verifying the prerequisites are met
                \item setting the limits requested by the step (see \python{CommandBase.get_limits()})
                \item evaluating the \python{DependentExpr} expressions
                \item calling the \python{run_command()} method
                \item checking the postconditions
                \item restoring the old limits
            \end{itemize}
        }

\end{itemize}

\section*{\python{LocalComputer}}\label{sec:LocalComputer}

\begin{itemize}[label={}]
    \item \python{recognized_limits = []}

    \item \docfunc{run_command(self, command, stdin, stdout, stderr, env)}
        \docfuncdesc{
            Runs the command using the \mintinline{shell}{/usr/bin/time} tool to measure the time and memory used.
            Returns the TODO \python{LocalComputer.LocalStats} object containing execution statistics.
        }
\end{itemize}

\section*{\python{KolejkaObserver}}\label{sec:KolejkaObserver}

\begin{itemize}[label={}]
    \item \python{recognized_limits = ['cpus', 'cpus_offset', 'pids', 'memory']}
\end{itemize}

\section*{\docfunc{detect_environment()}}\label{sec:detect_environment}

\begin{itemize}[label={}]
    \item Returns the environment based on the command line arguments.
    \item \mintinline{shell}{--local} (\textit{default}) -- \hyperref[sec:LocalComputer]{\python{LocalComputer}}
    \item \mintinline{shell}{--kolejka} -- \hyperref[sec:KolejkaObserver]{\python{KolejkaObserver}}
\end{itemize}
